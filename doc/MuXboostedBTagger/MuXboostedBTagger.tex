\documentclass{article}

%\usepackage{graphicx}% Include figure files
%\usepackage{subfig}
%\usepackage{wasysym} %wavy symbols (e.g. \apprge)

\newcommand{\muXmod}{\texttt{MuXboostedBTagger}}
\newcommand{\delphes}{\textsc{Delphes}}

\newcommand{\pT}{p_{T}^{}}

\newcommand{\pSubjet}[1][] {p_{\mathrm{subjet}}^{#1}}
\newcommand{\pCore}[1][]   {p_{\mathrm{core}}^{#1}}
\newcommand{\pCoreV}       {\vec{p}_{\mathrm{core}}^{}}
\newcommand{\pMu}[1][]     {p_{\mu}^{#1}}
\newcommand{\pMuV}       {\vec{p}_{\mu}^{}}
\newcommand{\pNu}[1][]     {p_{\nu_{\mu}}^{#1}}

\newcommand{\mB}             {m_{B}^{}}
\newcommand{\mCore}[1][]     {m_{\mathrm{core}}^{#1}}
\newcommand{\ECore}[1][]     {E_{\mathrm{core}}^{#1}}
\newcommand{\gammaCore}[1][] {\gamma_{\mathrm{core}}^{#1}}
\newcommand{\EMu}[1][]       {E_{\mu}^{#1}}

\begin{document}

\title{{\muXmod} user guide}

\author{Keith~Pedersen (kpeders1@hawk.iit.edu)}

\date{November 30, 2015}

\maketitle

\begin{abstract}
This user guide explains how to use the {\muXmod} module for {\delphes}, along with 
some of its internal kinematic expressions.
\end{abstract}

\section{Code notes}
\subsection{Robust subjet mass}
We define the subjet of $B$~hadron decay
%
\begin{equation}
\pSubjet=\pCore+\pMu+\pNu,
\end{equation}
%
where the ``core'' is ostensibly a charm hadron. 
Since the neutrino is not observable, we estimate it using the 
simplest choice ($\pNu=\pMu$, since most of the momentum comes from
their shared boost). This gives us an approximate subjet
%
\begin{equation}
\pSubjet\approx\pCore+2\,\pMu.
\end{equation}
%

The core is found by reclustering the jet using a much smaller radius 
parameter. This produces a list of \emph{candidate} cores, of which only
one is the ``correct'' core; this is found by using the hardest muon in 
the jet to find the core which produces $\sqrt{\pSubjet[2]}$ closest to 
$\mB=5.3$~GeV (the mass of a $B$~hadron admixture). 
However, since $\mCore$ is not accurately measured 
(from the granularity of the Calorimeter), we must first constrain it to its 
hypothesized mass under the $B$~hadron decay hypothesis ($\mCore=2$~GeV).

In order to find the core quickly, we don't want to correct the mass 
of each core candidate before adding the muon. Instead, we can use our
large cut on muon $\pT$ to treat the muon as massless, then calculate
the mass of the subjet analytically
%
\begin{equation}
\pSubjet[2]\approx\mCore[2]+2\,\pCore\cdot\pMu\approx\mCore[2]+4\,\EMu\ECore(1-\cos(\xi)\sqrt{1-g}),
\end{equation}
%
where $\xi$ is the angle between the muon and the core and $g\equiv\gammaCore[-2]$.
Using $y=\tan(\xi)^2=\left(\frac{\pCoreV\times\pMuV}{\pCoreV\cdot\pMuV}\right)^2$ 
and $\cos(\xi)=\frac{1}{\sqrt{1+y}}$ (since $\xi<\pi/2$), this becomes
%
\begin{equation}
\pSubjet[2]\approx\mCore[2]+4\,\EMu\ECore(1-\frac{\sqrt{1-g}}{\sqrt{1+y}}).
\end{equation}
%

This expression can be optimized to minimize floating point cancellation,
(which should be small, eliminating unnecessary
 cancellation is a good habit)
%
\begin{equation}
\pSubjet[2]\approx\mCore[2]+4\,\EMu\ECore\frac{g+y}{1+y+\sqrt{1-((g-y)+g\,y)}}.
\end{equation}

\end{document}
